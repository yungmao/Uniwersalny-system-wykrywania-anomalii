\section{Rozpatrywane podejścia}
Mając na uwadze problematykę detekcji anomalii poruszoną w rozdziale \ref{chap:OD} na fazie projekcyjnej rozpatrywano początkowo wybór jednego algorytmu o jak najlepszej skuteczności detekcji dla jak największej ilości zróżnicowanych zbiorów danych. 

Początkowy wyborem był algorytm \textit{Isolation Forest} \cite{iforest}. Algorytm pozytywnie oceniany był w badaniach porównawczych \cite{emmott2015meta,aggarwal2017ens}. Jednakże obecnie żaden algorytm konsekwentnie przewyższa -- skutecznością detekcji anomalii -- inne porównywane algorytmy \cite{aggarwal2017ens}. Dla przykładu w bardzo skupionych anomaliach algorytm ABOD \cite{abod} oraz LOF \cite{lof} sprawowały się lepiej \cite{emmott2015meta}. 

Następnie zamiast wyboru jednego algorytmu, rozpatrywano podejście zakładające wybór jak najlepszego algorytmu z listy algorytmów oraz dobór jego parametrów w celu usprawnienia detekcji anomalii dla zróżnicowanych właściwości zbiorów danych. 
W celu automatyzacji doboru optymalnego algorytmu oraz jego parametrów zdecydowano na wykorzystanie biblioteki \textit{Automating Outlier Detection via Meta-Learning} (MetaOD) \cite{zhao2020metaod}. 

\section{Proponowane rozwiązanie: MetaOD}
\label{sec:MetaOD}
MetaOD wykorzystuje meta-uczenie do systematycznego i kryterialnego podejścia automatyzacji wyboru optymalnego algorytmu wraz z jego parametrami (wybór modelu) w celu detekcji anomalii. Meta-uczenie, czyli uczenie jak się uczyć, jest techniką, która na podstawie wcześniejszych doświadczeń ułatwia efektywne uczenie dla nowego zadania (zbioru danych). Podejście jest skuteczne dla automatyzacji uczenia maszynowego \cite{vanschoren2018meta}. MetaOD rozpatruje problem wyboru modelu (algorytm wraz z parametrami) z przestrzeni bazowej modeli (Tabela \ref{tab:my_label}). MetaOD rozważa zadanie wyboru modelu w celu detekcji anomalii analogicznie
% OD podobnie jak CF i korzystając z badań na ten temat inny loss function
do problemu generacji rekomendacji (Collaborative Filtering) dla nowego użytkownika -- brak ewaluacji (problem "zimnego rozruchu"). Uwzględniając różnice między problemami m.in. brak sprzężenia zwrotnego do poprawy rekomendacji oraz wykorzystanie tylko jednego najlepszego modelu dla problemu detekcji anomalii zamiast Top-N rekomendacji.

MetaOD generuje na podstawie zbioru danych meta-dane (200 meta-cech), które reprezentują charakterystykę oraz właściwości zbioru danych m.in. miarę rozkładu. Generacja meta-cech przeprowadzany jest offline, natomiast wybór modelu z przestrzeni bazowej dokonywany jest online. MetaOD zwraca uporządkowaną listę modeli, od najlepszego do najgorszego, na podstawie zgromadzonych w bazie wyników skuteczności detekcji anomalii na zbiorach danych uznanych -- na podstawie meta-danych -- za podobne, z wykorzystaniem przestrzeni bazowej modeli.

W analizie skuteczności przeprowadzonych przez autorów: ,,MetaOD znacząco poprawił skuteczność detekcji w porównaniu z popularnymi metodami."\cite{zhao2020metaod}.



\begin{table}[]
    \centering
    \begin{tabular}{c|c|c}
    Algorytm & Parametr 1 & Parametr 2 \\ \hline
    ABOD     & liczba sąsiadów & nie dotyczy \\
    COF & liczba sąsiadów & nie dotyczy \\
    HBOS & liczba prostokątów histogramu & tolerancja \\
    IForest & liczba estymatorów & procent rozpatrywanych cech \\
    kNN & liczba sąsiadów & metoda \\
    LODA & liczba prostokątów histogramu & liczba losowych cięć \\
    LOF & liczba sąsiadów & metryka \\
    OCSVM &współczynnik $\nu$ & funkcja jądra
    \end{tabular}
    \caption{Przestrzeń bazowa modeli (algorytm i parametry) MetaOD}
        \footnotesize{źródło: Opracowanie własne na podstawie \cite{zhao2020metaod}}
    \label{tab:my_label}
\end{table}