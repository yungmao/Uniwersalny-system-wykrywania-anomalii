\section{Rozpatrywane podejścia}
Mając na uwadze problematykę detekcji anomalii poruszoną w rozdziale \ref{chap:OD} na fazie projekcyjnej rozpatrywano początkowo wybór jednego algorytmu o jak najlepszej skuteczności detekcji dla jak największej ilości zbiorów danych. 

Początkowy wyborem był algorytm \textit{Isolation Forest}\cite{iforest}. Algorytm pozytywnie oceniany był w badaniach porównawczych \cite{emmott2015meta,aggarwal2017ens}. Jednakże żaden algorytm konsekwentnie przewyższa -- precyzją detekcji anomalii -- inne porównywane algorytmy \cite{aggarwal2017ens}. Dla przykładu w bardzo skupionych anomaliach algorytm ABOD \cite{abod} oraz LOF \cite{lof} sprawowały się lepiej \cite{emmott2015meta}. 

Następnie zamiast wyboru jednego algorytmu, rozpatrywano podejście zakładające wybór jak najlepszego algorytmu z listy algorytmów oraz dobór jego parametrów w celu usprawnienia detekcji anomalii dla zróżnicowanych właściwości zbiorów danych. 
W celu automatyzacji doboru optymalnego algorytmu oraz jego parametrów zdecydowano na wykorzystanie biblioteki \textit{Automating Outlier Detection via Meta-Learning"} (MetaOD) \cite{zhao2020metaod}. 

\section{Proponowane rozwiązanie: MetaOD}
MetaOD wykorzystuje meta-uczenie do kryterialnego podejścia do zautomatyzowania wyboru optymalnego algorytmu wraz z jego parametrami(modelu) w celu detekcji anomalii. Meta-uczenie czyli uczenie jak się uczyć, jest techniką, która na podstawie wcześniejszych doświadczeń ułatwia efektywne uczenie dla nowego zadania (zbioru danych). Podejście jest skuteczne dla automatyzacji uczenia maszynowego \cite{vanschoren2018meta}. MetaOD rozpatruje problem doboru modelu (algorytm wraz z parametrami) analogicznie do problemu generacji rekomendacji (Collaborative Filtering) dla nowego użytkownika -- brak ewaluacji( problem "zimnego rozruchu"). Różnica między problemami to m.in. brak sprzężenia zwrotnego do poprawy rekomendacji oraz wybór tylko jednego najlepszego modelu dla problemu detekcji anomalii.

MetaOD do poprawnego działania generuje meta-cechy, które reprezentują charakterystykę oraz właściwości zbioru danych. 

\begin{table}[]
    \centering
    \begin{tabular}{c|c|c}
    Algorytm & Parametr 1 & Parametr 2 \\ \hline
    ABOD     & liczba sąsiadów & nie dotyczy \\
    COF & liczba sąsiadów & nie dotyczy \\
    HBOS & liczba prostokątów histogramu & tolerancja \\
    IForest & liczba estymatorów & procent rozpatrywanych cech \\
    kNN & liczba sąsiadów & metoda \\
    LODA & liczba prostokątów histogramu & liczba losowych cięć \\
    LOF & liczba sąsiadów & metryka \\
    OCSVM &współczynnik $\nu$ & funkcja jądra
    \end{tabular}
    \caption{Przestrzeń bazowa modeli (algorytm i parametry) MetaOD \cite{zhao2020metaod}}
    \label{tab:my_label}
\end{table}