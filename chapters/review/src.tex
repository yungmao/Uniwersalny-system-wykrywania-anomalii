\chapter{Sposób oceny pracy inżynierskiej}
\label{chap:review}

Praca inżynierska jest oceniana przez dwie osoby: opiekuna (mnie) oraz recenzenta. Jeśli wystawione przez nas oceny będą się znacząco różnić, zostaje wdrożona dodatkowa procedura. Niemniej, tego typu przypadki nie są częste.

Ocenie podlegają następujące elementy:
\begin{itemize}
    \item Wykonane urządzenie, program, etc.:
    \begin{itemize}
        \item stopień realizacji celu pracy;
        \item dobór metod i narzędzi realizacji pracy;
        \item innowacyjność rozwiązania (ocena wkładu własnego);
        \item jakość wykonania.
    \end{itemize}
    \item Dokumentacja pracy:
    \begin{itemize}
        \item kompletność;
        \item struktura logiczna tekstu;
        \item strona językowa;
        \item poziom edycji.
    \end{itemize}
\end{itemize}
W przypadku oceny niedostatecznej lub celującej, konieczne jest napisanie przez osobę oceniającą uzasadnienia. Poszczególne elementy składowe oceny nie mają żadnych ustalonych wag, wszystkie mają jednak znaczenie.