\section{Analiza technologiczna}
\subsection{Wykorzystane technologie oraz biblioteki}
System powstał z wykorzystaniem języka programistycznego Python 3.7. Jest to prosty do nauczenia, przejrzysty, a zarazem wszechstronny język programistyczny o rosnącej popularności zwłaszcza w środowisku uczenia maszynowego. Do stworzenia aplikacji webowej wykorzystano mikro-framework Flask 1.1.2. Do projektowania struktury i interfejsu graficznego strony internetowej wykorzystano: HTML, JavaScript oraz biblioteki CSS -- Bootstrap.
Do wyboru optymalnego modelu detekcji anomalii wykorzystano bibliotekę MetaOD, którą opisano w sekcji \ref{sec:MetaOD}. Do detekcji anomalii w zbiorze danych, wykorzystując algorytmy z przestrzeni bazowej modeli, skorzystano z gotowych implementacji algorytmów zawartych w bibliotece PyOD. 
Do analizy oraz przechowywania danych w strukturach danych użyto pakietu Pandas (tabela danych) oraz Numpy(tablice). Do skalowania zbioru danych wykorzystano bibliotekę Scikit-learn. Matplotlib został wykorzystany jako kreator wykresów pudełkowych oraz histogramów.

% Funkcjonalność aplikacji webowej została napisana w języku Python 3.7 z wykorzystaniem frameworku Flask 1.1.2.
% Interfejs graficzny powstała z użyciem języków HTML oraz JavaScript.
% Interfejs graficzny wykorzystuje bibliotekę CSS -- Bootstrap. 
% Do przechowywania danych w strukturach danych wykorzystano biblioteki: NumPy(tablice) oraz Pandas(tabela danych). 
% Detekcje anomalii przeprowadzona została z wykorzystaniem MetaOD i PyOD. 
% Do tworzenia wykresów wykorzystano bibloteki Matplotlib
% % \begin{itemize}
% %     \item Python 3.7 - 
% %     \item HTML
% %     \item Flask
% %     \item Bootstrap - bibloteka CSS, służąca do tworzenia interfejsu graficznego strony internetowej. 
% %     % \item Jinja2 
% %     \item PyOD
% %     \item MetaOD
% %     \item Sklearn
% %     \item FPDF
% %     \item Matplotlib
% %     \item Pandas
% %     \item Numpy
% % \end{itemize}
\subsection{Wykorzystane narzędzia programistyczne}
Zintegrowanym środowiskiem programistycznym wykorzystanym przy tworzeniu aplikacji był PyCharm Professional, czeskiej firmy JetBrains. To zaawansowane i wszechstronne środowisko ułatwia proces pisania kodu źródłowego, testowania oraz rozwijania oprogramowania. Posiada graficzny debugger ułatwiający identyfikację błędów. Wspiera pisanie aplikacji webowych. Posiada integrację z systemem kontroli wersji Git, wykorzystanego przy produkcji systemu do tworzenia kolejnych wersji rozwijanego projektu. Dzięki czemu implementacja nowych funkcjonalności odbywa się bez obawy utraty działającej wersji systemu.
