Celem niniejszej pracy inżynierskiej było zaprojektowanie oraz stworzenie systemu, który zapewniłby użytkownikowi na dokonanie analizy zbioru danych pod kątem występowania anomalii. Ideą kierującą projektowanie systemu było stworzenie swojego rodzaju ,,czarnej skrzynki'', która automatyzowałaby proces detekcji anomalii, dzięki czemu system byłby przystępny dla użytkownika bez wiedzy z zakresu detekcji anomalii. Problematyką zadania detekcji anomalii jest mnogość podejść w zależności od zastosowania, różnorodność danych. Brak możliwości ewaluacji na zbiorach danych bez etykiet znacząco utrudnia wybór jednego skutecznego modelu, który skutecznie wykrywałby anomalii w zbiorze danych.


W tradycyjnym podejściu detekcji anomalii wybór modelu kierowany był wstępną wiedzą na temat zbioru danych. Posiadanych cechy obserwacji, metodę generującą obserwacje. Prezentowany w niniejszej pracy system skutecznie wykrywa anomalie dla zróżnicowanych zbiorów danych bez wcześniejszej wiedzy na temat samego zbioru. System wykorzystuje w procesie doboru modelu detektora anomalii meta-uczenie. Dzięki czemu system na podstawie meta-cech uzyskanych z analizy zbioru danych rekomenduje modelu, zapewniając skuteczne zastosowanie systemu dla zróżnicowanych zbiorów danych.

Niniejsza praca oprócz samego stworzenia systemu jest też zwięzłym wprowadzeniem do tematu detekcji anomalii wraz z opisem wykorzystywanych algorytmów, dzięki czemu może być wykorzystana jako wstęp polskiego czytelnika w metody oraz postępów w zakresie detekcji anomalii. Tematu niszowego w polskiej literaturze naukowej. Jak również prezentuje innowacyjne podejście automatyzacji doboru detektora anomalii z wykorzystaniem biblioteki MetaOD.

Jednakże obecny stan techniki detekcji anomalii powoduje, że stworzenie uniwersalnie idealnego systemu jest niemożliwe. Wyniki anomalności wyznaczony przez system należy traktować jako wskaźnik, nad którymi obserwacjami należy przeprowadzić dalszą analizę w celu uzyskania wartościowej informacji. Przenosząc na użytkownika decyzję czy dana obserwacja jest, czy też nie jest anomalią, jednakże zapewniając informacje ułatwiające tę decyzję.

Stworzony system nie jest definitywnym rozwiązaniem problemu detekcji anomalii, charakter pracy skupiony jest na demonstracji meta-uczenia w zadaniu detekcji anomalii implementując go w stworzonym systemie w celu automatyzacji wyboru modelu detektora. Co w teorii zapewnia wszechstronność i uniwersalność systemu przez dostosowanie rozwiązania do problemu. Dodatkowo wraz z rozwojem biblioteki MetaOD, przez zwiększenie bazy historycznych wyników detekcji anomalii dla podobnych zbiorów, skuteczność systemu będzie rosła. 

