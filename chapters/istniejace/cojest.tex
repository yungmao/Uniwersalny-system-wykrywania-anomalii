\section{Definicja anomalii (obserwacji odstającej)}
Obserwacje odstające to punkty, które nie odpowiadają wzorcowi w zbiorze danych. Barnett i Lewis definiują obserwację odstającą następująco: ,,Obserwacja odstająca jest to obserwacja, której obecność jest różna od pozostałych obserwacji'' \cite{barnett1984outliers}. Rysunek \ref{fig:anomalia} obrazuje przykład obserwacji odstających (anomalii) dla dwuwymiarowego zbioru danych. Zbiór danych posiada obszar {$N_1$}, większość punktów znajduje się wewnątrz tego obszaru. Punkty wystarczająco oddalone od obszaru $N_1$: $o_1$, $o_2$, $o_3$ -- sklasyfikowane są jako obserwacje odstające (anomalie).
\begin{figure}[h]
    \centering
    \includegraphics[width=.7\textwidth]{chapters/istniejace/images/anomalia.png}
    \caption{Prosty przykład anomalii w dwuwymiarowym zbiorze danych.}
    \footnotesize{źródło: Opracowanie własne}
    \label{fig:anomalia}
\end{figure}

\section{Zadanie detekcji anomalii}
W pracy rozważamy nienadzorowane zadanie detekcji anomalii na zbiorze $N$ punktów $x_1,...,x_N$ każdy punkt jest d-wymiarowym wektorem liczb rzeczywistych. Zbiór danych składa się z obserwacji poprawnych i anomalnych, jednakże zbiór nie posiada wektora $y_1,..,y_N$ -- klasyfikującego obserwację do obserwacji poprawnych lub anomalnych. Celem zadania detekcji anomalii jest wykrycie punktów anomalnych w danym zbiorze.

\section{Różnice w podejściu detekcji anomalii}

\subsection{Wykrywanie anomalii lokalnych a globalnych}
\label{sub:loc_glob}
Podejście dotyczy wyboru zakresu zbioru jako zbioru odniesienia dla rozpatrywania odstawania danej obserwacji. Główne podejścia to podejście globalne oraz lokalne. Rysunek \ref{fig:anomalie_glob_lok} obrazuje prosty dwuwymiarowy zbiór danych z dwoma klastrami (skupieniami) poprawnych obserwacji: $c_1$ i $c_2$. Dla punktów $x_1 $ oraz $x_2$ klasyfikacja jako anomalie jest możliwa wizualnie. Oba punkty są znacząco oddalone od gęstych obszarów obserwacji. Są to anomalie globalne. Rozpatrując wszystkie obserwacje zbioru punkt $x_3$ mógłby być uznany za poprawną obserwację należącą do klastra $c_2$, jednak kiedy rozpatrzymy podzbiór punktów w sąsiedztwie klastra $c_2$ (zbiór odniesienia), punkt $x_3$ może być uznany za anomalię. Jest to przykład anomalii lokalnej. Zatem anomalia lokalna jest to obserwacja, której odchylenie od normy (anomalność) rozpatrywane jest w obszarze najbliższego sąsiedztwa (zbioru odniesienia). 
\begin{itemize}
    \item Podejście globalne 
    \begin{itemize}
        \item Zbiór odniesienia obejmuje wszystkie obserwacje
        \item Założenie o istnieniu jeden prawidłowego mechanizmu generującego normalne punkty.
        \item Problem: występowanie anomalii lokalnych lub więcej niż jeden prawidłowy mechanizm generujący punkty mogą wypaczyć wynik detekcji
    \end{itemize}
    \item Podejście lokalne 
    \begin{itemize}
        \item Zbiór odniesienia jest podzbiorem całego zbioru
        \item Brak założenia o liczbie mechanizmów generujących.
        \item Problem: wybór odpowiedniego podzbioru jako zbioru odniesienia
    \end{itemize}
\end{itemize}

\begin{figure}[h]
    \centering
    \includegraphics[width=.65\textwidth]{chapters/istniejace/images/mikro_cluste.png}
    \caption{Przykład anomalii globalnych ($x_1, x_2$), lokalnej $x_3$ oraz mikro klastra $c_3$. }
    \footnotesize{źródło: Opracowanie własne na podstawie \cite{goldstein2016comparative}}
    \label{fig:anomalie_glob_lok}
\end{figure}

\subsection{Wynik metody}
\label{sec:score}
Ważnym aspektem dla zadania wykrywania anomalii jest sposób oceny każdej obserwacji. Dane wyjściowe metody mogą przypisywać obserwacji wartość wskaźnika mierzącego anomalność na dwa warianty:
\begin{itemize}
    \item binarny: przypisanie obserwacji etykiety. Klasyfikacja obserwacji jako anomalii lub poprawnej obserwacji w zbiorze
    \item ciągły: dla każdej obserwacji obliczana jest wartość anomalności np. prawdopodobieństwo obserwacji jako anomalii
\end{itemize}
Podejście przypisujące wartość wskaźnika mierzącego anomalność w sposób ciągły (wartość anomalności) jest podejściem wszechstronnym. Wykorzystując podejście osoba analizująca dane może za pomocą np. progu wartości anomalności, otrzymać binarną etykietę obserwacji, dostosowując próg dla danej dziedziny.
Wiele podejść opartych na przypisywaniu wartości anomalności skupia się na wyznaczeniu grupy n-obserwacji o najwyższym wyniku (parametr n często podawany jest przez użytkownika np. procent kontaminacji zbioru danych). 
% Podejście oparte na ciągłym wyniku anomalności punktu przydatne jest w 
Podejście przypisujące wartość anomalności przydatne jest w rozważaniu mikro klastrów, czyli małych regularnych klastrów. Rysunek \ref{fig:anomalie_glob_lok} obrazuje mikro klaster $c_3$. Algorytm detekcji anomalii powinien przypisać punktom klastra wartość anomalności wyższą od normalnych obserwacji oraz mniejszą od anomalii globalnych. Przykład pokazuje, że podejście przypisywania ciągłej wartości anomalności punktu jest podejściem bardziej użytecznym od binarnej klasyfikacji zwłaszcza w dalszej analizie danych. 

\subsection{Rodzaje anomalii}
% Punktowe, collective i contextual - można wszystko z punktowych które są dominujące w unsupervised detection
W detekcji anomalii wyróżnia się trzy rodzaje anomalii: anomalie punktowe, anomalie zgrupowane oraz anomalie kontekstowe.
Większość dostępnych algorytmów detekcji anomalii dla uczenia nienadzorowanego opiera się na wykrywaniu anomalii punktowych. Detekcja anomalii zgrupowanych została przedstawiona na rysunku \ref{fig:anomalie_glob_lok} i opisana w sekcji \ref{sec:score} -- mikro klaster $c_3$.
Anomalie kontekstowe są to anomalie najczęściej spotykane w szeregach czasowych. Dotyczy to punktów, które mogą być uznane za poprawne obserwacje, jednak rozpatrywane w kontekście charakteru zbioru danych zostaną uznane za anomalie. Weźmy przykład pomiaru temperatury w skali roku we Wrocławiu. Temperatura 25$^{\circ}$C wydaje się poprawnym odczytem temperatury, jednak w kontekście miesiąca np. stycznia. Tak wysoka temperatura jest anomalią.

Szczęśliwie można wykorzystać algorytmy detekcji anomalii punktowych do wykrywania anomalii zgrupowanych i kontekstowych. 
Detekcja różnych typów anomalii wymaga obróbki danych m.in. przekształcenia zbioru danych reprezentującego cechy z wykorzystaniem korelacji, agregacji oraz grupowania \cite{goldstein2014behavior} w celu analizy obserwacji jako anomalii punktowych.

W pracy skupiono się na problemach detekcji anomalii punktowych, wykorzystując bazę zbiorów danych ODDS \cite{ODDS}. Baza zawiera zbiory danych już przygotowanych do zadania detekcji anomalii punktowych.


