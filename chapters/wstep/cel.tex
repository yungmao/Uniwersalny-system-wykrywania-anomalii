\section{Cel pracy}
Celem pracy jest opracowanie oraz implementacja uniwersalnego systemu internetowego służącego do detekcji anomalii dla dowolnych zbiorów danych obserwacji statystycznych (bez etykiet). System ma pozwalać na przesłanie danych w popularnych formatach CSV oraz JSON. Przetworzeniu uzyskanych danych m.in. oczyszczeniu danych z brakujących wartości oraz normalizacji danych. Stworzeniu raportu z działania systemu oraz zapewnieniu użytkownikowi oczyszczonych danych z anomalii w formacie, w którym te dane zostały przesłane. System powinien być jak najprostszy w obsłudze, szybki w działaniu oraz zapewniać kluczowe informacje pozwalającemu użytkownikowi na dalszą analizę uzyskanych wyników. 
System ma zapewnić użytkownikowi nieposiadającemu wiedzy na temat wykrywania anomalii automatyzację procesu wyboru algorytmu detekcji oraz jego parametrów w celu zwiększenia skuteczności detekcji.


Praca składa się pięciu rozdziałów. Rozdział pierwszy ma za zadanie wprowadzić czytelnika w tematykę zagadnienia. Rozdział drugi dogłębnie przedstawić aktualne metody na detekcje anomalii. Rozdział trzeci poświęcony jest omówieniu stworzonego systemu. Rozdział czwarty jest dogłębną analizą poprawności działania systemu. Ostatni rozdział podsumowuje całość pracy oraz prezentuje możliwości rozwoju systemu. 