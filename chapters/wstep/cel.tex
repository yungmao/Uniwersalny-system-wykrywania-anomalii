\section{Cel pracy}
Celem pracy jest opracowanie oraz implementacja uniwersalnego systemu internetowego służącego do detekcji anomalii dla dowolnych zbiorów danych obserwacji statystycznych (bez etykiet). System ma pozwalać na przesłanie danych w popularnych formatach CSV oraz JSON. Przetworzeniu uzyskanych danych m.in. oczyszczeniu danych z brakujących wartości oraz wyskalowaniu danych. Stworzeniu raportu z działania systemu oraz zapewnieniu użytkownikowi oczyszczonych danych z anomalii w formacie, w którym te dane zostały przesłane, jak również danych z wartością anomalności obserwacji -- jak bardzo dana obserwacja odstaje od reszty zbioru. System powinien być jak najprostszy w obsłudze, szybki w działaniu oraz zapewniać kluczowe informacje pozwalającemu użytkownikowi na dalszą analizę uzyskanych wyników. 
System ma zapewnić użytkownikowi, nieposiadającemu wiedzy na temat wykrywania anomalii, automatyzację procesu wyboru algorytmu detekcji oraz jego parametrów w celu zwiększenia skuteczności detekcji.

\section{Struktura pracy}

Praca składa się sześciu rozdziałów. Rozdział pierwszy wprowadza czytelnika w tematykę pracy. Rozdział drugi przedstawia teorię detekcje anomalii wraz z różnicami w podejściu i problematyce tematu. Rozdział trzeci prezentuje zastosowane podejście doboru modelu oraz opisuje algorytmy wraz z parametrami (modele) składające się na przestrzeń bazową modeli.  Rozdział czwarty dokonuje omówienia stworzonego systemu wykrywania anomalii. Rozdział piąty analizuje skuteczność działania systemu. Ostatni rozdział podsumowuje całość pracy. 