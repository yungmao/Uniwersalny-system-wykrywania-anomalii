\section{Wprowadzenie}
\label{sec:wprowadzenie}

Yuval Noah Harari w swojej książce \textit{,,21 lekcji dla XXI wieku''} zwiastuje potencjał \textit{Big data}\footnote{Gromadzenie i przetwarzanie dużych zbiorów danych w celu uzyskania wartościowych informacji} w XXI wieku jako dobra, której wartość swoim potencjałem przyćmi tradycyjne dobra takie jak posiadanie ziemi, fabryk czy machin.
,,Harvard Business Review'' określa pozycję \textit{Data Scientist}\footnote{Specjalista analizujący dane w celu uzyskania pożądanych i wartościowych informacji} jako najbardziej pożądanej w XXI wieku \cite{davenport2012data}. Znaczenie eksploracji danych zwiększa się każdego dnia.
Ważną dziedziną tego procesu jest zadanie wykrywania anomalii. Początki badań nad metodą wykrywania anomalii w danych można znaleźć już w XIX w. \cite{edgeworth1887xli}. 
Wraz z rozwojem techniki komputerowej oraz zwiększoną ilością dostępnych danych, skuteczne metody wykrywania anomalii stały się pożądane. 
Detekcja anomalii w danych przekłada się na uzyskaniu cennych informacji w wielu dziedzinach takich jak:
\begin{itemize}
    \item W diagnostyce medycznej anomalny obraz z rezonansu magnetycznego może wskazać obecność nowotworu \cite{spence2001detection}
    \item W detekcji oszust finansowych takich jak kradzież karty kredytowej \cite{bolton2001unsupervised}
    \item W przemyśle do detekcji awarii sensorów \cite{dereszynski2011spatiotemporal} 
    \item W zabezpieczeniach sieciowych do wykrywania włamań do sieci \cite{garcia2009anomaly}
\end{itemize} 

Wykrywanie anomalii odnosi się do problemu znajdowania obserwacji, które odbiegają zachowaniem od normy. Słownik języka polskiego definiuje anomalię jako: ,,odchylenie od normy'' \cite{pwn}. 
W literaturze można spotkać się z określeniami takich punktów jako: nieprawidłowość, obserwacja odstająca, niezgodność, dewiacja. \cite{aggarwal2017outlier}. Z czego najczęstszym określeniem w danym kontekście jest anomalia oraz obserwacja odstająca, które czasami używane są zamiennie. Związane jest to z podejściem zakładającym, że problem wykrywania anomalii jest uczeniem nienadzorowanym. Do detekcji anomalii bazujemy na metodach statystycznych w celu znalezieniu elementów odstających z założeniem, że owe punkty będą anomaliami. Jest to dominujące założenie w literaturze \cite{emmott2015meta}. Drugim podejściem jest modelowanie osobno normalnych i anomalnych punktów. Jednakże, to podejście wymaga znajomości procesu powstania obu typów punktów w zbiorze danych lub wystarczającą liczbę oznaczonych danych szkoleniowych. Co w problemach wymagających detekcji anomalii, gdzie operujemy na nieoznaczonych danych oraz nie znamy procesu generującego anomalne punkty, jest podejściem nieprzydatnym. W związku z tym w niniejszej pracy do detekcji punktów anomalnych wykorzystamy podejście bazujące na detekcji elementów odstających.
% Wykrywanie anomalii odnosi się do problemu znajdywania punktów, które odbiegają zachowaniem od spodziewanego wzoru. Takie punkty nazywane są anomaliami, obserwacjami odstającymi, wyjątkami, niespodziankami. W literaturze najczęściej wykorzystuje się zwroty anomalia oraz obserwacja odstająca. Zwroty te często używane są zamiennie. 