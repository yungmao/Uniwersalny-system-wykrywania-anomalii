\chapter{Praca z \LaTeX{}-em}
\label{chap:latex}

Podstawowe informacje dotyczące pracy z systemem \LaTeX{} można znaleźć na stronie Overleaf~\cite{Overleaf}. Z kolei szczegółową specyfikację dotyczącą konkretnych paczek zawarto na repozytorium~\cite{WinNT}.

\section{Przykładowa struktura projektu}

W niniejszym dokumencie zaproponowano szablon projektu pokazany na rysunku~\ref{fig:projectSchema}. Większość elementów ma oczywiste znaczenie, poniżej objaśniono wybrane:
\begin{itemize}
    \item alg --- folder z kodem dla pseudokodów;
    \item cmd --- folder z makrami wykorzystywanymi w rozdziale;
    \item data --- folder z danymi wykorzystywanymi do generowania wykresów lub tabel;
    \item fig --- folder z obrazami;
    \item tab --- folder z kodem generującym tabele;
    \item tikz --- folder z kodem tikz, służącym na przykład do generowania schematów lub wykresów;
    \item global.tex --- plik z makrami używanymi w całym projekcie.
\end{itemize}

\begin{figure}[t]
    \centering
    \begin{minipage}{0.45\textwidth}
        \subimport{tikz/}{filetree.tex}
    \end{minipage}
    \begin{minipage}{0.45\textwidth}
        \subimport{tikz/}{chaptertree.tex}
    \end{minipage}
    \caption{Przykładowa struktura projektu: ogólna (po lewo) oraz struktura folderów \textit{appendices} i \textit{chapters} (po prawo)}
    \label{fig:projectSchema}
\end{figure}

\section{Przydatne funkcje edytora Overleaf}

Oto wybrane przydatne funkcje edytora Overleaf, z którymi warto się moim zdaniem zapoznać:
\begin{itemize}
    \item dedykowane skróty klawiszowe;
    \item możliwość dodawania komentarzy w trybie ,,Review'' --- w ten sposób możemy wymieniać uwagi dotyczące tekstu;
    \item chat --- do szybkiej wymiany uwag;
    \item historia --- w szczególności istotne jest dodawanie nazwanych wersji, pozwalają one na przywrócenie w razie potrzeby dokumentu do używalności;
    \item synchronizacja dokumentu z kontem Dropbox oraz udostępnienie go jako repozytorium Git --- opcja przydatna w razie chęci pracy offline.
\end{itemize}