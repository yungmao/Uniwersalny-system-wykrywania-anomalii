\chapter{Struktura pracy inżynierskiej}%
\label{chap:structure}

Struktura pracy inżynierskiej powinna być dostosowana do tematu i skonsultowana z opiekunem. Typową pracę można podzielić na kilka części, wymienionych poniżej:
\begin{description}[
    leftmargin = !,
    labelwidth = 1em,
]
    \item[]   \textbf{Spis treści}
    \item[]   \textbf{Lista symboli} --- opcjonalny spis wszystkich używanych symboli matematycznych (wraz z objaśnieniami), obowiązujących w całej pracy; przydatny tylko w przypadku silnie zmatematyzowanych prac.
    \item[]   \textbf{Lista skrótów} --- opcjonalny spis wszystkich używanych skrótów (wraz z objaśnieniami), obowiązujący w całej pracy; przydatny tylko jeśli w pracy występuje bardzo dużo słabo znanych skrótów.
    \item[1.] \textbf{Wstęp} --- słowo wstępne, umiejscawiające temat pracy inżynierskiej we współczesnym świecie; tekst ten powinien znaleźć się bezpośrednio pod tytułem rozdziału, przed kolejnymi sekcjami.
    \begin{description}
        \item[1.1.] \textbf{Cel pracy} --- Zarysowanie problemu którego praca (oraz, zwykle, towarzyszący jej wytwór pracy inżynierskiej), po którym w punktach opisuje się poszczególne elementy które mają być efektem pracy; cele te powinny być mierzalne.
        \item[1.2.] \textbf{Zakres pracy} --- krótki opis zawartości kolejnych rozdziałów pracy.
    \end{description}
    \item[2.] \textbf{Przegląd technologii i literatury} --- jeden lub kilka rozdziałów, w których przedstawiany jest stan zastany wiedzy dotyczącej rozwiązywanego problemu, wraz z opisem wykorzystanych technologii.
    \item[3.] \textbf{Opis zaproponowanego rozwiązania} --- jeden lub kilka rozdziałów stanowiących opis opracowanego rozwiązania, wraz ze szczegółowym przedstawieniem procesu jego opracowania (od projektu, aż do gotowego ,,produktu''); treść i zakres zależy w dużym stopniu od tematu.
    \item[4.] \textbf{Aplikacje/Badania eksperymentalne} --- jeden lub kilka rozdziałów stanowiących opis wykorzystania opracowanego rozwiązania, wraz z jego testami i ewaluacją; treść i zakres zależy w dużym stopniu od tematu.
    \item[5.] \textbf{Wnioski i uwagi} --- rozdział w którym przestawiane są wnioski wynikające z przeprowadzonych badań/opracowanych rozwiązań, wraz z propozycją dalszych prac.
    \item[]   \textbf{Dodatki} --- wydzielona część pracy, w której znajdują się dodatkowe materiały, których umieszczenie w głównym ciele pracy zaburzyłoby jej strukturę i utrudniło czytanie; mogą się tam na przykład znaleźć wycinki kodu źródłowego aplikacji, plany płytek PCB, dodatkowe tabele i wykresy prezentujące wyniki badań, obszerne dowody twierdzeń, instrukcja obsługi stworzonej aplikacji, załączniki, etc..
    \item[]   \textbf{Bibliografia} --- spis literatury, do której odnośniki znajdują się w tekście; lądują tutaj również w większości wypadków wszelkie linki do stron \texttt{www}.
    \item[]   \textbf{Spis rysunków} --- opcjonalny spis rysunków, generowany automatycznie przez \LaTeX{}, ma sens jedynie w przypadku obszernej pracy.
    \item[]   \textbf{Spis tabel} --- j/w.
    \item[]   \textbf{Spis algorytmów} --- j/w.
    \item[]   \textbf{Indeks} --- opcjonalny spis słów kluczowych, wraz z numerami stron na których wystąpiły, generowany półautomatycznie przez \LaTeX{}, występuje właściwie tylko w książkach (powyżej 200 stron).
    \item[]   \textbf{Glosariusz} --- opcjonalny spis ważnych pojęć, wraz z ich objaśnieniami, występuje właściwie tylko w książkach (powyżej 200 stron).   
\end{description}